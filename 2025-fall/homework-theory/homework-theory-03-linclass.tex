\documentclass[12pt,fleqn]{article}
\usepackage{vkCourseML}
\usepackage{cancel}
%\usepackage{vkCourseML}
\hypersetup{unicode=true}
%\usepackage[a4paper]{geometry}
\usepackage[hyphenbreaks]{breakurl}

\theorembodyfont{\rmfamily}
\newtheorem{esProblem}{Задача}

\begin{document}

\title{Машинное обучение\\ФКН ВШЭ\\Теоретическое домашнее задание №3}

\date{}

\author{}

\maketitle


\textbf{Задача 1.} Бандерлог из Лога\footnote{деревня в Кадуйском районе Вологодской области} ведёт свой блог, любит считать логарифмы и оценивать логистические регрессии. С помощью нового классификатора $b(x)$, предсказывающего оценку принадлежности объекта положительному классу, Бандерлог решил задачу классификации на 8 объектах. Предсказания $b(x)$ и реальные метки объектов приведены ниже:

$$
\begin{array}{lc}
b\left(x_{1}\right)=0.1, & y_{1}=+1, \\
b\left(x_{2}\right)=0.8, & y_{2}=+1, \\
b\left(x_{3}\right)=0.2, & y_{3}=-1, \\
b\left(x_{4}\right)=0.25, & y_{4}=-1, \\
b\left(x_{5}\right)=0.9, & y_{5}=+1, \\
b\left(x_{6}\right)=0.3, & y_{6}=+1, \\
b\left(x_{7}\right)=0.6, & y_{7}=-1, \\
b\left(x_{8}\right)=0.8, & y_{8}=-1, \\
b\left(x_{9}\right)=0.95, & y_{8}=+1 .
\end{array}
$$

\begin{enumerate}
  \item Постройте ROC-кривую и вычислите AUC-ROC для множества классификаторов $a(x ; t)$, порожденных $b(x)$, на выборке $X$.
  \item Постройте PR-кривую и найдите площадь под ней для того же множества классификаторов.
  \item Как по-английски будет «бревно»?
\end{enumerate}

\textbf{Задача 2.} Пусть дан классификатор $b(x)$, который возвращает оценку принадлежности объекта $x$ положительному классу. Отсортируем все объекты по неубыванию ответа классификатора: $b\left(x_{(1)}\right) \leq \cdots \leq b\left(x_{(\ell)}\right)$. Обозначим истинные ответы на этих объектах через $y_{(1)}, \ldots, y_{(\ell)}$.

\begin{enumerate}
  \item Покажите, что AUC-ROC для данной выборки будет равен вероятности того, что случайно выбранный положительный объект окажется в отсортированном списке не раньше случайно выбранного отрицательного объекта.
  \item Покажите, что число дефектных пар в выборке $y_{(1)}, \ldots, y_{(\ell)}$ будет совпадать с числом итераций, которые нужно сделать для того, чтобы отсортировать этот массив с помощью сортировки пузырьком.
\end{enumerate}

\textbf{Задача 3.} Пусть дана некоторая выборка $X$ и классификатор $b(x)$, возвращающий в качестве оценки принадлежности объекта $x$ положительному классу 0 или 1 (а не некоторое вещественное число, как предполагалось на семинарах).

\begin{enumerate}
  \item Постройте ROC-кривую для классификатора $b(x)$ на выборке $X$.
  \item Покажите, что AUC-ROC классификатора $b(x)$ на выборке $X$ может быть выражен через долю правильных ответов и полноту классификатора $a(x ; t)$, получающегося при выборе некоторого порога $t \in(0 ; 1)$. Помимо указанных величин в формулу могут входить только величины $\ell_{-}, \ell_{+}, \ell$ (количество отрицательных, положительных и общее количество объектов в выборке $X$ соответственно).
  \item Покажите, что в случае сбалансированной выборки ( $\ell_{-}=\ell_{+}$) AUC-ROC классификатора $b(x)$ на выборке $X$ совпадает с долей правильных ответов классификатора при выборе некоторого порога $t \in(0 ; 1)$.
\end{enumerate}

\textbf{Задача 4.} Вам даны предикты классификаторов a(x), b(x) и верные метки объектов:

$$
\begin{array}{lc}
b\left(x_{1}\right)=0.9, a\left(x_{1}\right)=0.99, & y_{1}=+1, \\
b\left(x_{2}\right)=0.85, a\left(x_{1}\right)=0.96, & y_{1}=+1, \\
b\left(x_{3}\right)=0.8, a\left(x_{1}\right)=0.40, & y_{1}=+1, \\b\left(x_{4}\right)=0.6, a\left(x_{1}\right)=0.35, & y_{1}=+1, \\b\left(x_{5}\right)=0.55, a\left(x_{1}\right)=0.95, & y_{1}=-1, \\
b\left(x_{6}\right)=0.5, a\left(x_{1}\right)=0.92, & y_{1}=-1, \\
b\left(x_{7}\right)=0.45, a\left(x_{1}\right)=0.50, & y_{1}=-1, \\b\left(x_{8}\right)=0.4, a\left(x_{1}\right)=0.45, & y_{1}=-1, \\
b\left(x_{9}\right)=0.35, a\left(x_{1}\right)=0.3, & y_{1}=-1, \\
b\left(x_{10}\right)=0.3, a\left(x_{1}\right)=0.25, & y_{1}=-1, \\
b\left(x_{11}\right)=0.2, a\left(x_{1}\right)=0.2, & y_{1}=-1, \\
b\left(x_{12}\right)=0.1, a\left(x_{1}\right)=0.1, & y_{1}=-1.
\end{array}
$$

\begin{enumerate}
    \item Чему равен AUC-ROC у моделей?
    \item Посчитайте precision, recall и F1 при пороге 0.9 у моделей, что вы видите?
    \item Какие пороги дают максимальную F1 для моделей? 
    \item Что показывает AUC-ROC с точки зрения ранжирования?
    \item Изменится ли результат, если добавить какие-либо активации к результатам модели без переобучения?
\end{enumerate}

\textbf{Задача 5.} В анализе данных для сравнения среднего значения некоторой величины у объектов двух выборок часто используется критерий Манна-Уитни-Уилкоксона\footnote{\href{https://en.wikipedia.org/wiki/Mann-Whitney_U_test}{https://en.wikipedia.org/wiki/Mann-Whitney\_U\_test}}, основанный на вычислении $U$-статистики.

Пусть у нас имеется выборка $X$ и классификатор $b(x)$, возвращающий оценку принадлежности объекта $x$ положительному классу. Тогда вычисление $U$-статистики для подвыборки $X$, состоящей из объектов положительного класса, производится следующим образом: объекты обеих выборок сортируются по неубыванию значения $b(x)$, после чего каждому объекту в полученном упорядоченном ряду $x_{(1)}, \ldots, x_{(\ell)}$ присваивается ранг - номер позиции $r_{(i)}$ в ряду (начиная с 1 , при этом для объектов с одинаковыми значением $b(x)$ в качестве ранга присваивается среднее значение ранга для таких объектов). Тогда $U$-статистика для объектов положительного класса равна:

$$
U_{+}=\sum_{\substack{i=1 \\ y_{(i)}=+1}}^{\ell} r_{(i)}-\frac{\ell_{+}\left(\ell_{+}+1\right)}{2}.
$$

Покажите, что для значения AUC-ROC классификатора $b(x)$ на выборке $X$ и $U$-статистики верно следующее соотношение:

$$
\mathrm{AUC}=\frac{U_{+}}{\ell_{-} \ell_{+}}.
$$

\textbf{Задача 6.} Женя хочет построить в этом новом мире атмосферу доброты и взаимоподдержки. Для этого он собирается обучить классификатор токсичных сообщений, чтобы тот их автоматически банил. В выборке Жени половина сообщений токсичные. В обучающей 9000 сообщений, в тестовой 1000 сообщений.

\begin{enumerate}
  \item В качестве порога для бана Женя взял 0.5. Точность классификатора (precision) получилась 0.9. Полнота (recall) получилась 0.7. Как выглядит матрица ошибок (confusion matrix)?
  \item На самом деле в чате токсик встречается только в $20 \%$ случаев. Женя натравил классификатор на 1000 свежих сообщений. Сколько из них классификатор назовёт токсичными? Сколько раз он ошибётся в этом? Как примерно будет выглядеть матрица ошибок?\\
  \item Айнура требует от Жени метрики качества его классификатора. Женя разметил 100 случайных сообщений, которые классификатор забанил и 100 случайных сообщений, которые классификатор не стал банить. По этой выборке он оценил precision, recall, accuracy и FPR. Он утверждает, что эти метрики отражают качество работы классификатора на потоке свежих сообщений.

Айнура не согласна с Женей и считает, что его оценки смещены из-за того, что они посчитаны на сбалансированной выборке, а токсика всего лишь $20 \%$. Правда ли это?\\
  \item Чтобы все метрики качества соответствовали потоку, можно было бы сделать случайную выборку из него. Однако Айнуре лень. Помогите ей вывести формулы, которые скорректируют оценки Жени.
\end{enumerate}

\end{document}