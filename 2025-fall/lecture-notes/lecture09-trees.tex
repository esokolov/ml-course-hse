\documentclass[12pt,fleqn]{article}
\usepackage{vkCourseML}
\hypersetup{unicode=true}
%\usepackage[a4paper]{geometry}
\usepackage[hyphenbreaks]{breakurl}

\interfootnotelinepenalty=10000

\begin{document}
\title{Лекция 7\\Решающие деревья}
\author{Е.\,А.\,Соколов\\ФКН ВШЭ}
\maketitle

Мы уделили достаточно много внимания линейным методам, которые обладают рядом важных достоинств:
быстро обучаются, способны работать с большим количеством объектов и признаков,
имеют небольшое количество параметров, легко регуляризуются.
При этом у них есть и серьёзный недостаток~--- они могут восстанавливать только
линейные зависимости между целевой переменной и признаками.
Конечно, можно добавлять в выборку новые признаки, которые нелинейно зависят от исходных,
но этот подход является чисто эвристическим, требует выбора типа нелинейности,
а также всё равно ограничивает сложность модели сложностью признаков~(например,
если признаки квадратичные, то и модель сможет восстанавливать только зависимости второго порядка).

В данной лекции мы разберёмся с решающими деревьями~--- семейством моделей, которые позволяют восстанавливать нелинейные зависимости
произвольной сложности.

Решающие деревья хорошо описывают процесс принятия решения во многих ситуациях.
Например, когда клиент приходит в банк и просит выдать ему кредит, то сотрудник банка
начинает проверять условия:
\begin{enumerate}
    \item Какой возраст у клиента? Если меньше 18, то отказываем в кредите, иначе продолжаем.
    \item Какая зарплата у клиента? Если меньше 50 тысяч рублей, то переходим к шагу 3, иначе к шагу 4.
    \item Какой стаж у клиента? Если меньше 10 лет, то не выдаем кредит, иначе выдаем.
    \item Есть ли у клиента другие кредиты? Если есть, то отказываем, иначе выдаем.
\end{enumerate}
Такой алгоритм, как и многие другие, очень хорошо описывается решающим деревом.
Это отчасти объясняет их популярность.

Первые работы по использованию решающих деревьев для анализа данных появились в 60-х годах,
и с тех пор несколько десятилетий им уделялось очень большое внимание.
Несмотря на свою интерпретируемость и высокую выразительную способность,
деревья крайне трудны для оптимизации из-за свой дискретной структуры~---
дерево нельзя продифференцировать по параметрам и найти с помощью градиентного спуска
хотя бы локальный оптимум.
Более того, даже число параметров у них не является постоянным и может меняться в зависимости
от глубины, выбора критериев дробления и прочих деталей.
Из-за этого все методы построения решающих деревьев являются жадными и эвристичными.

На сегодняшний день решающие деревья не очень часто используются как отдельные методы классификации или регрессии.
В то же время, как оказалось, они очень хорошо объединяются в композиции~--- решающие леса,
которые являются одними из наиболее сильных и универсальных моделей.

\section{Определение решающего дерева}
Рассмотрим бинарное дерево, в котором:
\begin{itemize}
    \item каждой внутренней вершине~$v$ приписана функция~(или предикат)~$\beta_v: \XX \to \{0, 1\}$;
    \item каждой листовой вершине~$v$ приписан прогноз~$c_v \in Y$~(в случае с классификацией
        листу также может быть приписан вектор вероятностей).
\end{itemize}
Рассмотрим теперь алгоритм~$a(x)$, который стартует из корневой вершины~$v_0$
и вычисляет значение функции~$\beta_{v_0}$.
Если оно равно нулю, то алгоритм переходит в левую дочернюю вершину, иначе в правую,
вычисляет значение предиката в новой вершине и делает переход или влево, или вправо.
Процесс продолжается, пока не будет достигнута листовая вершина;
алгоритм возвращает тот класс, который приписан этой вершине.
Такой алгоритм называется~\emph{бинарным решающим деревом}.

На практике в большинстве случаев используются одномерные предикаты~$\beta_v$,
которые сравнивают значение одного из признаков с порогом:
\[
    \beta_v(x; j, t)
    =
    [x_j < t].
\]
Существуют и многомерные предикаты, например:
\begin{itemize}
    \item линейные~$\beta_v(x) = [\langle w, x \rangle < t]$;
    \item метрические~$\beta_v(x) = [\rho(x, x_v) < t]$, где точка~$x_v$ является одним из объектов выборки любой точкой
        признакового пространства.
\end{itemize}
Многомерные предикаты позволяют строить ещё более сложные разделяющие поверхности,
но очень редко используются на практике~--- например, из-за того,
что усиливают и без того выдающиеся способности деревьев к переобучению.
Далее мы будем говорить только об одномерных предикатах.

\section{Построение деревьев}
Легко убедиться, что для любой выборки можно построить решающее дерево,
не допускающее на ней ни одной ошибки~--- даже с простыми одномерными предикатами
можно сформировать дерево, в каждом листе которого находится ровно по одному объекту выборки.
Скорее всего, это дерево будет переобученным и не сможет показать хорошее качество
на новых данных.
Можно было бы поставить задачу поиска дерева, которое является минимальным~(с точки зрения количества листьев)
среди всех деревьев, не допускающих ошибок на обучении~--- в этом случае можно было бы
надеяться на наличие у дерева обобщающей способности.
К сожалению, эта задача является NP-полной,
и поэтому приходится ограничиваться жадными алгоритмами построения дерева.

Опишем базовый жадный алгоритм построения бинарного решающего дерева.
Начнем со всей обучающей выборки~$X$ и найдем наилучшее ее разбиение
на две части~$R_1(j, t) = \{x \cond x_j < t\}$ и~$R_2(j, t) = \{x \cond x_j \geq t\}$
с точки зрения заранее заданного функционала качества~$Q(X, j, t)$.
Найдя наилучшие значения~$j$ и~$t$, создадим корневую вершину дерева,
поставив ей в соответствие предикат~$[x_j < t]$.
Объекты разобьются на две части~--- одни попадут в левое поддерево, другие в правое.
Для каждой из этих подвыборок рекурсивно повторим процедуру, построив дочерние вершины для корневой,
и так далее.
В каждой вершине мы проверяем, не выполнилось ли некоторое условие останова~--- и если выполнилось,
то прекращаем рекурсию и объявляем эту вершину листом.
Когда дерево построено, каждому листу ставится в соответствие ответ.
В случае с классификацией это может быть класс, к которому относится больше всего
объектов в листе, или вектор вероятностей~(скажем, вероятность класса может
быть равна доле его объектов в листе).
Для регрессии это может быть среднее значение, медиана или другая функция от целевых переменных
объектов в листе.
Выбор конкретной функции зависит от функционала качества в исходной задаче.

Решающие деревья могут обрабатывать пропущенные значения~--- ситуации, в которых для некоторых
объектов неизвестны значения одного или нескольких признаков.
Для этого необходимо модифицировать процедуру разбиения выборки в вершине,
что можно сделать несколькими способами.

После того, как дерево построено, можно провести его~\emph{стрижку}~(pruning)~---
удаление некоторых вершин с целью понижения сложности и повышения обобщающей способности.
Существует несколько подходов к стрижке, о которых мы немного упомянем ниже.

Таким образом, конкретный метод построения решающего дерева определяется:
\begin{enumerate}
    \item Видом предикатов в вершинах;
    \item Функционалом качества~$Q(X, j, t)$;
    \item Критерием останова;
    \item Методом обработки пропущенных значений;
    \item Методом стрижки.
\end{enumerate}
Также могут иметь место различные расширения, связанные с учетом весов объектов,
работой с категориальными признакам и т.д.
Ниже мы обсудим варианты каждого из перечисленных пунктов.

\section{Критерии информативности}
При построении дерева необходимо задать~\emph{функционал качества},
на основе которого осуществляется разбиение выборки на каждом шаге.
Обозначим через~$R_m$ множество объектов, попавших в вершину, разбиваемую на данном шаге,
а через~$R_\ell$ и~$R_r$~--- объекты, попадающие в левое и правое поддерево соответственно
при заданном предикате.
Мы будем использовать функционалы следующего вида:
\[
    Q(R_m, j, s)
    =
    H(R_m)
    -
    \frac{|R_\ell|}{|R_m|}
    H(R_\ell)
    -
    \frac{|R_r|}{|R_m|}
    H(R_r).
\]
Здесь~$H(R)$~--- это~\emph{критерий информативности}~(impurity criterion),
который оценивает качество распределения целевой переменной среди объектов множества~$R$.
Чем меньше разнообразие целевой переменной, тем меньше должно быть значение критерия информативности~---
и, соответственно, мы будем пытаться минимизировать его значение.
Функционал качества~$Q(R_m, j, s)$ мы при этом будем максимизировать.

Как уже обсуждалось выше, в каждом листе дерево будет выдавать константу~--- вещественное число, вероятность
или класс.
Исходя из этого, можно предложить оценивать качество множества объектов~$R$ тем,
насколько хорошо их целевые переменные предсказываются константой~(при оптимальном выборе этой константы):
\[
    H(R)
    =
    \min_{c \in \YY}
    \frac{1}{|R|}
    \sum_{(x_i, y_i) \in R}
        L(y_i, c),
\]
где~$L(y, c)$~--- некоторая функция потерь.
Далее мы обсудим, какие именно критерии информативности часто используют в задачах регрессии и классификации.

\subsection{Регрессия}
Как обычно, в регрессии выберем квадрат отклонения в качестве функции потерь.
В этом случае критерий информативности будет выглядеть как
\[
    H(R)
    =
    \min_{c \in \YY}
    \frac{1}{|R|}
    \sum_{(x_i, y_i) \in R}
        (y_i - c)^2.
\]
Как известно, минимум в этом выражении будет достигаться на среднем значении целевой переменной.
Значит, критерий можно переписать в следующем виде:
\[
    H(R)
    =
    \frac{1}{|R|}
    \sum_{(x_i, y_i) \in R}
    \left(
        y_i
        -
        \frac{1}{|R|}
        \sum_{(x_j, y_j) \in R}
            y_j
    \right)^2.
\]
Мы получили, что информативность вершины измеряется её дисперсией~---
чем ниже разброс целевой переменной, тем лучше вершина.
Разумеется, можно использовать и другие функции ошибки~$L$~---
например, при выборе абсолютного отклонения мы получим в качестве критерия среднее абсолютное отклонение от медианы.

\subsection{Классификация}
Обозначим через~$p_{k}$ долю объектов класса~$k$~($k \in \{1, \dots, K\}$), попавших в вершину~$R$:
\[
    p_{k}
    =
    \frac{1}{|R|}
    \sum_{(x_i, y_i) \in R}
        [y_i = k].
\]
Через~$k_*$ обозначим класс, чьих представителей оказалось больше всего среди объектов,
попавших в данную вершину: $k_* = \argmax_k p_{k}$.

\subsubsection{Ошибка классификации}
Рассмотрим индикатор ошибки как функцию потерь:
\[
    H(R)
    =
    \min_{c \in \YY}
    \frac{1}{|R|}
    \sum_{(x_i, y_i) \in R}
        [y_i \neq c].
\]
Легко видеть, что оптимальным предсказанием тут будет наиболее популярный класс~$k_*$~---
значит, критерий будет равен следующей доле ошибок:
\[
    H(R)
    =
    \frac{1}{|R|}
    \sum_{(x_i, y_i) \in R}
        [y_i \neq k_*]
    =
    1 - p_{k_*}.
\]

Данный критерий является достаточно грубым,
поскольку учитывает частоту~$p_{k_*}$ лишь одного класса.

\subsubsection{Критерий Джини}
Рассмотрим ситуацию, в которой мы выдаём в вершине не один класс,
а распределение на всех классах~$c = (c_1, \dots, c_K)$, $\sum_{k = 1}^{K} c_k = 1$.
Качество такого распределения можно измерять, например, с помощью критерия Бриера~(Brier score):
\[
    H(R)
    =
    \min_{\sum_k c_k = 1}
    \frac{1}{|R|}
    \sum_{(x_i, y_i) \in R}
    \sum_{k = 1}^{K}
        (c_k - [y_i = k])^2.
\]

Можно показать, что оптимальный вектор вероятностей состоит из долей классов~$p_k$:
\[
    c_* = (p_1, \dots, p_K)
\]
Если подставить эти вероятности в исходный критерий информативности
и провести ряд преобразований, то мы получим критерий~Джини:
\[
    H(R)
    =
    \sum_{k = 1}^{K}
        p_k (1 - p_k).
\]

\subsubsection{Энтропийный критерий}
Мы уже знакомы с более популярным способом оценивания качества
вероятностей~--- логарифмическими потерями, или логарифмом правдоподобия:
\[
    H(R)
    =
    \min_{\sum_k c_k = 1} \left(
        -
        \frac{1}{|R|}
        \sum_{(x_i, y_i) \in R}
        \sum_{k = 1}^{K}
            [y_i = k]
            \log c_k
    \right).
\]
Для вывода оптимальных значений~$c_k$ вспомним, что все значения~$c_k$
должны суммироваться в единицу.
Как известного из методов оптимизации, для учёта этого ограничения необходимо искать
минимум лагранжиана:
\[
    L(c, \lambda)
    =
    -
    \frac{1}{|R|}
    \sum_{(x_i, y_i) \in R}
    \sum_{k = 1}^{K}
        [y_i = k]
        \log c_k
    +
    \lambda
    \sum_{k = 1}^{K}
        c_k
    \to
    \min_{c_k}
\]
Дифференцируя, получаем:
\[
    \frac{\partial}{\partial c_k}
    L(c, \lambda)
    =
    -
    \frac{1}{|R|}
    \sum_{(x_i, y_i) \in R}
        [y_i = k]
        \frac{1}{c_k}
    +
    \lambda
    =
    - \frac{p_k}{c_k}
    +
    \lambda
    =
    0,
\]
откуда выражаем~$c_k = p_k / \lambda$.
Суммируя эти равенства по~$k$, получим
\[
    1 = \sum_{k = 1}^{K} c_k = \frac{1}{\lambda} \sum_{k = 1}^{K} p_k = \frac{1}{\lambda},
\]
откуда~$\lambda = 1$.
Значит, минимум достигается при~$c_k = p_k$, как и в предыдущем случае.
Подставляя эти выражения в критерий, получим, что он будет представлять собой энтропию распределения классов:
\[
    H(R)
    =
    -
    \sum_{k = 1}^{K}
        p_k
        \log p_k.
\]

Из теории вероятностей известно, что энтропия ограничена снизу нулем, причем минимум достигается на вырожденных
распределениях~($p_i = 1$, $p_j = 0$ для~$i \neq j$).
Максимальное же значение энтропия принимает для равномерного распределения.
Отсюда видно, что энтропийный критерий отдает предпочтение более <<вырожденным>> распределениям классов
в вершине.

%\subsubsection{Выбор критерия}
%Рассмотрим простой пример с двумя классами.
%Пусть в текущую вершину попало~$400$ объектов первого класса
%и~$400$ объектов второго класса.
%Допустим, нужно сделать выбор между двумя разбиениями,
%одно из которых генерирует поддеревья с числом объектов~$(300, 100)$ и~$(100, 300)$~(первое
%число в паре~--- число объектов первого класса в подвыборке, второе~---число объектов второго
%класса),
%а другое~--- с числом объектов~$(200, 400)$ и~$(200, 0)$.
%Оба разбиения дают ошибку классификации~$0.25$,
%но критерий Джини и энтропийный критерий отдадут предпочтение
%второму разбиению, что логично,
%поскольку правая вершина окажется листовой и сложность дерева
%окажется меньше.

%В заключение отметим, что нет никаких четких правил для выбора функционала качества,
%и на практике лучше всего выбирать его с помощью кросс-валидации.


\section{Критерии останова}
Можно придумать большое количестве критериев останова.
Перечислим некоторые ограничения и критерии:
\begin{itemize}
    \item Ограничение максимальной глубины дерева.
    \item Ограничение минимального числа объектов в листе.
    \item Ограничение максимального количества листьев в дереве.
    \item Останов в случае, если все объекты в листе относятся к одному классу.
    \item Требование, что функционал качества при дроблении улучшался как минимум на~$s$ процентов.
\end{itemize}

С помощью грамотного выбора подобных критериев и их параметров можно существенно повлиять
на качество дерева.
Тем не менее, такой подбор является трудозатратным и требует проведения кросс-валидации.

\section{Методы стрижки дерева}
Стрижка дерева является альтернативой критериям останова, описанным выше.
При использовании стрижки сначала строится переобученное дерево~(например, до тех пор, пока
в каждом листе не окажется по одному объекту), а затем производится оптимизация его структуры
с целью улучшения обобщающей способности.
Существует ряд исследований, показывающих, что стрижка позволяет достичь лучшего качества
по сравнению с ранним остановом построения дерева на основе различных критериев.

Тем не менее, на данный момент методы стрижки редко используются
и не реализованы в большинстве библиотек для анализа данных.
Причина заключается в том, что деревья сами по себе являются слабыми алгоритмами и не представляют большого интереса,
а при использовании в композициях они либо~\emph{должны} быть переобучены~(в случайных лесах),
либо должны иметь очень небольшую глубину~(в бустинге), из-за чего необходимость в стрижке отпадает.

Одним из методов стрижки является~\emph{cost-complexity pruning}.
Обозначим дерево, полученное в результате работы жадного алгоритма, через~$T_0$.
Поскольку в каждом из листьев находятся объекты только одного класса,
значение функционала~$R(T)$ будет минимально на самом дереве~$T_0$~(среди всех поддеревьев).
Однако данный функционал характеризует лишь качество дерева на обучающей выборке,
и чрезмерная подгонка под нее может привести к переобучению.
Чтобы преодолеть эту проблему, введем новый функционал~$R_\alpha(T)$,
представляющий собой сумму исходного функционала~$R(T)$ и штрафа за размер дерева:
\begin{equation}
\label{eq:pruningFunc}
    R_\alpha(T) = R(T) + \alpha |T|,
\end{equation}
где~$|T|$~--- число листьев в поддереве~$T$, а~$\alpha \geq 0$~--- параметр.
Это один из примеров~\emph{регуляризованных} критериев качества,
которые ищут баланс между качеством классификации обучающей выборки
и сложностью построенной модели.

Можно показать, что существует последовательность вложенных деревьев с одинаковыми корнями:
\[
    T_K \subset T_{K-1} \subset \dots \subset T_0,
\]
(здесь~$T_K$~--- тривиальное дерево, состоящее из корня дерева~$T_0$),
в которой каждое дерево~$T_i$ минимизирует критерий~\eqref{eq:pruningFunc}
для $\alpha$ из интервала $\alpha \in [\alpha_i, \alpha_{i+1})$,
причем
\[
    0 = \alpha_0 < \alpha_1 < \dots < \alpha_K < \infty.
\]
Эту последовательность можно достаточно эффективно найти путем обхода дерева.
Далее из нее выбирается оптимальное дерево по отложенной выборке или с помощью кросс-валидации.

\section{Обработка пропущенных значений}
Одним из основных преимуществ решающих деревьев является возможность работы
с пропущенными значениями.
Рассмотрим некоторые варианты.

% TODO: это все очень непонятно и должно быть переписано
Пусть нам нужно вычислить функционал качества для предиката~$\beta(x) = [x_j < t]$,
но в выборке~$R$ для некоторых объектов не известно значение признака~$j$~---
обозначим их через~$V_{j}$.
В таком случае при вычислении функционала можно просто проигнорировать эти объекты,
сделав поправку на потерю информации от этого:
\[
    Q(R, j, s)
    \approx
    \frac{|R \setminus V_{j}|}{|R|}
    Q(R \setminus V_{j}, j, s).
\]
Затем, если данный предикат окажется лучшим, поместим объекты из~$V_{j}$
как в левое, так и в правое поддерево.
Также можно присвоить им при этом веса~$|R_\ell| / |R|$ в левом поддереве и~$|R_r| / |R|$ в правом.
В дальнейшем веса можно учитывать, добавляя их как коэффициенты перед индикаторами~$[y_i = k]$
во всех формулах.

На этапе применения дерева необходимо выполнять похожий трюк.
Если объект попал в вершину, предикат которой не может быть вычислен из-за пропуска,
то прогнозы для него вычисляются в обоих поддеревьях, и затем усредняются
с весами, пропорциональными числу обучающих объектов в этих поддеревьях.
Иными словами, если прогноз вероятности для класса~$k$ в поддереве~$R_m$ обозначается через~$a_{mk}(x)$,
то получаем такую формулу:
\[
    a_{mk}(x)
    =
    \begin{cases}
        a_{\ell k}(x), \quad &\beta_m(x) = 0;\\
        a_{r k}(x), \quad &\beta_m(x) = 1;\\
        \frac{|R_\ell|}{|R_m|} a_{\ell k}(x)
            +
            \frac{|R_r|}{|R_m|} a_{r k}(x),
            \quad
            &\beta_m(x)\ \text{нельзя вычислить}.
    \end{cases}
\]

Другой подход заключается в построении~\emph{суррогатных предикатов} в каждой вершине.
Так называется предикат, который использует другой признак, но при этом дает
разбиение, максимально близкое к данному.

Отметим, что нередко схожее качество показывают и гораздо более простые способы обработки пропусков~---
например, можно заменить все пропуски на ноль.
Для деревьев также разумно будет заменить пропуски в признаке на числа,
которые превосходят любое значение данного признака.
В этом случае в дереве можно будет выбрать такое разбиение по этому признаку,
что все объекты с известными значениями пойдут в левое поддерево, а все объекты с пропусками~--- в правое.

\section{Учет категориальных признаков}
Самый очевидный способ обработки категориальных признаков~--- разбивать вершину на столько поддеревьев,
сколько имеется возможных значений у признака~(multi-way splits).
Такой подход может показывать хорошие результаты, но при этом есть риск получения дерева
с крайне большим числом листьев.

Рассмотрим подробнее другой подход.
Пусть категориальный признак~$x_j$ имеет множество значений~$Q = \{u_1, \dots, u_q\}$, $|Q| = q$.
Разобьем множество значений на два непересекающихся подмножества:~$Q = Q_1 \sqcup Q_2$,
и определим предикат как индикатор попадания в первое подмножество:~$\beta(x) = [x_j \in Q_1]$.
Таким образом, объект будет попадать в левое поддерево, если признак~$x_j$ попадает в множество~$Q_1$,
и в первое поддерево в противном случае.
Основная проблема заключается в том, что для построения оптимального предиката
нужно перебрать~$2^{q - 1} - 1$ вариантов разбиения,
что может быть не вполне возможным.

Оказывается, можно обойтись без полного перебора в случаях с бинарной
классификацией и регрессией~\cite{hastie09esl}.
Обозначим через~$R_m(u)$ множество объектов, которые попали в вершину~$m$
и у которых~$j$-й признак имеет значение~$u$;
через~$N_m(u)$ обозначим количество таких объектов.

В случае с бинарной классификацией упорядочим все значения категориального
признака на основе того, какая доля объектов с таким значением имеет класс~$+1$:
\[
    \frac{1}{N_m(u_{(1)})}
    \sum_{x_i \in R_m(u_{(1)})}
        [y_i = +1]
    \leq
    \dots
    \leq
    \frac{1}{N_m(u_{(q)})}
    \sum_{x_i \in R_m(u_{(q)})}
        [y_i = +1],
\]
после чего заменим категорию~$u_{(i)}$ на число~$i$,
и будем искать разбиение как для вещественного признака.
Можно показать, что если искать оптимальное разбиение по критерию Джини или энтропийному критерию,
то мы получим такое же разбиение, как и при переборе по всем возможным~$2^{q - 1} - 1$ вариантам.

Для задачи регрессии с MSE-функционалом это тоже будет верно, если упорядочивать
значения признака по среднему ответу объектов с таким значением:
\[
    \frac{1}{N_m(u_{(1)})}
    \sum_{x_i \in R_m(u_{(1)})}
        y_i
    \leq
    \dots
    \leq
    \frac{1}{N_m(u_{(q)})}
    \sum_{x_i \in R_m(u_{(q)})}
        y_i.
\]
Именно такой подход используется в библиотеке
Spark MLlib~\footnote{\url{http://spark.apache.org/docs/latest/mllib-decision-tree.html}}.

\section{Методы построения деревьев}
Существует несколько популярных методов построения деревьев:
\begin{itemize}
    \item ID3: использует энтропийный критерий. Строит дерево до тех пор,
        пока в каждом листе не окажутся объекты одного класса,
        либо пока разбиение вершины дает уменьшение энтропийного критерия.
    \item C4.5: использует критерий Gain Ratio~(нормированный энтропийный критерий).
        Критерий останова~--- ограничение на число объектов в листе.
        Стрижка производится с помощью метода Error-Based Pruning,
        который использует оценки обобщающей способности для принятия решения об удалении вершины.
        Обработка пропущенных значений осуществляется с помощью метода,
        который игнорирует объекты с пропущенными значениями при вычислении критерия ветвления,
        а затем переносит такие объекты в оба поддерева с определенными весами.
    \item CART: использует критерий Джини. Стрижка осуществляется с помощью Cost-Complexity Pruning.
        Для обработки пропусков используется метод суррогатных предикатов.
\end{itemize}

\section{Решающие деревья и линейные модели}
Как следует из определения, решающее дерево~$a(x)$ разбивает
всё признаковое пространство на некоторое количество непересекающихся подмножеств~$\{J_1, \dots, J_n\}$,
и в каждом подмножестве~$J_j$ выдаёт константный прогноз~$w_j$.
Значит, соответствующий алгоритм можно записать аналитически:
\[
    a(x)
    =
    \sum_{j = 1}^{n}
        w_j
        [x \in J_j].
\]
Обратим внимание, что это линейная модель над признаками~$([x \in J_j])_{j = 1}^{n}$~---
а ведь в начале лекции мы хотели избавиться от линейности!
Получается, что решающее дерево с помощью жадного алгоритма подбирает преобразование признаков
для данной задачи, а затем просто строит линейную модель над этими признаками.
Далее мы увидим, что многие нелинейные методы машинного обучения можно
представить как совокупность линейных методов и хитрых способов порождения признаков.

\begin{thebibliography}{1}
\bibitem{hastie09esl}
    \emph{Hastie T., Tibshirani R., Friedman J.} (2009).
    The Elements of Statistical Learning.
\end{thebibliography}
\end{document}
