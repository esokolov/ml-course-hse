\documentclass[12pt,fleqn]{article}

\usepackage{vkCourseML}

\usepackage{lipsum}
\usepackage{indentfirst}
\title{Машинное обучение, ФКН ВШЭ\\ Метрики классификации}
\author{}
\date{}
\theorembodyfont{\rmfamily}
\newtheorem{esProblem}{Задача}
\begin{document}

\maketitle


\begin{esProblem}
    Бандерлог из Лога\footnote{деревня в Кадуйском районе Вологодской области} ведёт свой блог, любит считать логарифмы и оценивать логистические регрессии. С помощью нового классификатора~$b(x)$, предсказывающего оценку принадлежности объекта положительному классу, Бандерлог решил задачу классификации на $8$ объектах. Предсказания~$b(x)$ и реальные метки объектов приведены ниже:
	\begin{align*}
		&b(x_1) = 0.1, \quad  y_1 = +1,\\
		&b(x_2) = 0.8, \quad y_2 = +1,\\
		&b(x_3) = 0.2, \quad y_3 = -1,\\
		&b(x_4) = 0.25, \quad y_4 = -1,\\
		&b(x_5) = 0.9, \quad y_5 = +1,\\
		&b(x_6) = 0.3, \quad y_6 = +1,\\
		&b(x_7) = 0.6, \quad y_7 = -1,\\
		&b(x_8) = 0.95, \quad y_8 = +1.\\
	\end{align*}
	
    \begin{enumerate} 
        \item Постройте ROC-кривую и вычислите AUC-ROC для множества классификаторов~$a(x;t)$, порожденных~$b(x)$, на выборке~$X$.
        \item Постройте PR-кривую и найдите площадь под ней для того же множества классификаторов. 
        \item Как по-английски будет «бревно»?
    \end{enumerate} 
\end{esProblem}

\begin{esProblem}
	Пусть дан  классификатор $b(x)$, который возвращает оценку принадлежности объекта $x$ положительному классу. Отсортируем все объекты по неубыванию ответа классификатора: $b(x_{(1)}) \le \dots \le b(x_{(\ell)})$. Обозначим истинные ответы на этих объектах через $y_{(1)}, \dots, y_{(\ell)}$.
	
	\begin{enumerate} 
	\item Покажите, что AUC-ROC для данной выборки будет равен вероятности того, что случайно выбранный положительный объект окажется в отсортированном списке не раньше случайно выбранного отрицательного объекта.
	
	\item Покажите, что число дефектных пар в выборке $y_{(1)}, \dots, y_{(\ell)}$ будет совпадать с числом итераций, которые нужно сделать для того, чтобы отсортировать этот массив с помощью сортировки пузырьком. 
	\end{enumerate} 
\end{esProblem}

\begin{esProblem}
	Пусть дана некоторая выборка $X$ и классификатор $b(x),$ возвращающий в качестве оценки принадлежности объекта  $x$ положительному классу 0 или 1 (а не некоторое вещественное число, как предполагалось на семинарах).
	\begin{enumerate}
	\item Постройте ROC-кривую для классификатора $b(x)$ на выборке $X$.
	\item Покажите, что AUC-ROC классификатора $b(x)$ на выборке $X$ может быть выражен через долю правильных ответов и полноту классификатора $a(x; t)$, получающегося при выборе некоторого порога $t \in (0; 1)$. Помимо указанных величин в формулу могут входить только величины $\ell_-, \, \ell_+, \, \ell$ (количество отрицательных, положительных и общее количество объектов в выборке  $X$ соответственно).
	\item Покажите, что в случае сбалансированной выборки ($\ell_- = \ell_+$) AUC-ROC классификатора $b(x)$ на выборке $X$ совпадает с долей правильных ответов классификатора при выборе некоторого порога $t \in (0;1).$
	\end{enumerate} 
\end{esProblem}


\begin{esProblem}
	В анализе данных для сравнения среднего значения некоторой величины у объектов двух выборок часто используется критерий Манна–Уитни–Уилкоксона\footnote{\href{https://en.wikipedia.org/wiki/Mann\%2dWhitney\_U\_test}{https://en.wikipedia.org/wiki/Mann–Whitney\_U\_test}}, основанный на вычислении $U$-статистики.
	\par Пусть у нас имеется выборка $X$ и классификатор $b(x)$, возвращающий оценку принадлежности объекта $x$ положительному классу. Тогда вычисление $U$-статистики для подвыборки $X$, состоящей из объектов положительного класса, производится следующим образом: объекты обеих выборок сортируются по неубыванию значения~$b(x)$, после чего каждому объекту в полученном упорядоченном ряду $x_{(1)}, \dots, x_{(\ell)}$ присваивается ранг — номер позиции $r_{(i)}$ в ряду (начиная с 1, при этом для объектов с одинаковыми значением $b(x)$ в качестве ранга присваивается среднее значение ранга для таких объектов). Тогда $U$-статистика для объектов положительного класса равна:
	$$U_+ = \sum_{\substack{i= 1 \\ y_{(i)} = +1}}^\ell r_{(i)} - \frac{\ell_+ (\ell_+ + 1)}{2}.$$
	\par Покажите, что для значения AUC-ROC классификатора $b(x)$ на выборке $X$ и $U$-статистики верно следующее соотношение:
	$$\text{AUC} = \frac{U_+}{\ell_- \ell_+}.$$
\end{esProblem}


\begin{esProblem}
Женя хочет построить в этом новом мире атмосферу доброты и взаимоподдержки. Для этого он собирается обучить классификатор токсичных сообщений, чтобы тот их автоматически банил. В выборке Жени половина сообщений токсичные. В обучающей $9000$ сообщений, в тестовой $1000$ сообщений. 

\begin{enumerate} 
    \item В качестве порога для бана Женя взял $0.5$. Точность  классификатора (precision) получилась $0.9$. Полнота (recall) получилась $0.7$. Как выглядит матрица ошибок (confusion matrix)? 
    
    \item На самом деле в чате токсик встречается только в $20\%$ случаев. Женя натравил классификатор на $1000$ свежих сообщений. Сколько из них классификатор назовёт токсичными? Сколько раз он ошибётся в этом? Как примерно будет выглядеть матрица ошибок? 
    
    \item Айнура требует от Жени метрики качества его классификатора. Женя разметил $100$ случайных сообщений, которые классификатор забанил и $100$ случайных сообщений, которые классифкатор не стал банить. По этой выборке он оценил precision, recall, accuracy и FPR. Он утверждает, что эти метрики отражают качество работы классифкатора на потоке свежих сообщений.  
    
    Айнура не согласна с Женей и считает, что его оценки смещены из-за того, что они посчитаны на сбалансированной выборке, а токсика всего лишь $20\%$. Правда ли это?
    
    \item Чтобы все метрики качества соответствовали потоку, можно было бы сделать случайную выборку из него. Однако Айнуре лень. Помогите ей вывести формулы, которые скорректируют оценки Жени.
\end{enumerate} 
\end{esProblem}

\end{document}

