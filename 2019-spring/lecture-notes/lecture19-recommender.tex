\documentclass[12pt,fleqn]{article}
\usepackage{vkCourseML}
\hypersetup{unicode=true}
%\usepackage[a4paper]{geometry}
\usepackage[hyphenbreaks]{breakurl}

\interfootnotelinepenalty=10000

\begin{document}
\title{Лекция 19\\Рекомендательные системы}
\author{Е.\,А.\,Соколов\\ФКН ВШЭ}
\maketitle

Задача рекомендательной системы~--- выбрать для пользователя элементы из некоторого множества,
на которых он с наибольшей вероятностью совершит интересные нам действия.
Это могут быть:
\begin{itemize}
    \item товары, которые пользователь захочет купить;
    \item музыка, которую пользователь захочет дослушать до конца;
    \item статьи, которые пользователь дочитает до конца;
    \item видео, которые пользователь досмотрит до конца;
    \item и многое другое.
\end{itemize}

Мы будем рассуждать в терминах пользователей~(users, $U$) и товаров~(items, $I$),
но все методы подходят для рекомендаций любых объектов.
Будем считать, что для некоторых пар пользователей~$u \in U$ и товаров~$i \in I$ известны
оценки~$r_{ui}$, которые отражают степень заинтересованности пользователя в товаре.
Вычисление таких оценок~--- отдельная тема.
Например, в интернет-магазине заинтересованность может складываться из покупок товара
и просмотров его страницы, причём покупки должны учитываться с большим весом.
В социальной сети заинтресованность в материале может складываться из времени просмотра,
кликов и явного отклика~(лайки, репосты); это всё тоже должно суммироваться с различными весами.
Не будем сейчас останавливаться на этом вопросе, а перейдём к основной задаче.

Требуется по известным рейтингам~$r_{ui}$ научиться строить для каждого пользователя~$u$
набор из~$k$ товаров~$I(u)$, наиболее подходящих данному пользователю~--- то есть таких,
для которых рейтинг~$r_{ui}$ окажется максимальным.

Получается понятная задача: для объекта~$x_{ui}$~<<пользователь-товар>> нужно предсказать значение
целевой переменной~$r_{ui}$.
Здесь требуют уточнения три пункта: целевая переменная, признаки и функционал ошибки.
Первое мы затронули выше, третье обсудим позже, а сейчас поговорим о том,
какими признаками можно охарактеризовать пару~<<пользователь-товар>>.

\section{Признаки в рекомендательных системах}

\subsection{Коллаборативная фильтрация}
Как понять, что пользователю может понравиться товар?
Первый вариант~--- поискать похожих на него пользователей и посмотреть, что нравится им;
также можно поискать товары, похожие на те, которые этот пользователь уже покупал.
Методы коллаборативной фильтрации строят рекомендации для пользователя на основе
похожестей между пользователями и товарами.
Мы рассмотрим два подхода к определению сходства.

\subsubsection{Memory-based}
Два пользователя похожи, если они ставят товарам одинаковые оценки.
Рассмотрим двух пользователей~$u$ и~$v$
Обозначим через~$I_{uv}$ множество товаров~$i$, для которых известны
оценки обоих пользователей:
\[
    I_{uv}
    =
    \{
        i \in I
        \cond
        \exists r_{ui}
        \ \&\ 
        \exists r_{vi}
    \}.
\]
Тогда сходство двух данных пользователей можно вычислить через корреляцию Пирсона:
\[
    w_{uv}
    =
    \frac{
        \sum_{i \in I_{uv}}
            (r_{ui} - \bar r_u)
            (r_{vi} - \bar r_v)
    }{
        \sqrt{
        \sum_{i \in I_{uv}}
            (r_{ui} - \bar r_u)^2
        }
        \sqrt{
        \sum_{i \in I_{uv}}
            (r_{vi} - \bar r_v)^2
        }
    },
\]
где~$\bar r_u$ и~$\bar r_v$~--- средние рейтинги пользователей по множеству товаров~$I_{uv}$.

Чтобы вычислять сходства между товарами~$i$ и~$j$, введём множество пользователей~$U_{ij}$,
для которых известны рейтинги этих товаров:
\[
    U_{ij}
    =
    \{
        u \in U
        \cond
        \exists r_{ui}
        \ \&\ 
        \exists r_{uj}
    \}.
\]
Тогда сходство двух данных товаров можно вычислить через корреляцию Пирсона:
\[
    w_{ij}
    =
    \frac{
        \sum_{u \in U_{ij}}
            (r_{ui} - \bar r_i)
            (r_{uj} - \bar r_j)
    }{
        \sqrt{
        \sum_{u \in U_{ij}}
            (r_{ui} - \bar r_i)^2
        }
        \sqrt{
        \sum_{u \in U_{ij}}
            (r_{uj} - \bar r_j)^2
        }
    },
\]
где~$\bar r_i$ и~$\bar r_j$~--- средние рейтинги товаров по множеству пользователей~$U_{ij}$.
Отметим, что существуют и другие способы вычисления похожестей~---
например, можно вычислять скалярные произведения между векторами рейтингов двух товаров.

Мы научились вычислять сходства товаров и пользователей~---
разберём теперь несколько способов определения товаров, которые стоит
рекомендовать пользователю~$u_0$.
В подходе на основе сходств пользователей~(user-based collaborative filtering)
определяется множество~$U(u_0)$ пользователей, похожих на данного:
\[
    U(u_0)
    =
    \{v \in U
        \cond
        w_{u_0 v} > \alpha
    \}.
\]
После этого для каждого товара вычисляется, как часто он покупался пользователями из~$U(u_0)$:
\[
    p_{i}
    =
    \frac{
        |\{u \in U(u_0) \cond \exists r_{ui}\}|
    }{
        |U(u_0)|
    }.
\]
Пользователю рекомендуются~$k$ товаров с наибольшими значениями~$p_i$.
Данный подход позволяет строить рекомендации, если для данного пользователя найдутся похожие.
Если же пользователь является нетипичным, то подобрать что-либо не получится.

Также существует подход на основе сходств товаров~(item-based collaborative filtering).
В нём определяется множество товаров, похожих на те, которые интересовали данного пользователя:
\[
    I(u_0)
    =
    \{
        i \in I
        \cond
        \exists r_{u_0 i_0},
        w_{i_0 i} > \alpha
    \}.
\]
Затем для каждого товара из этого множества вычисляется его сходство с пользователем:
\[
    p_i
    =
    \max_{i_0: \exists r_{u_0 i_0}}
    w_{i_0 i}.
\]
Пользователю рекомендуются~$k$ товаров с наибольшими значениями~$p_i$.
Даже если пользователь нетипичный, то данный подход может найти товары, похожие
на интересные ему~--- и для этого необязательно иметь пользователя со схожими интересами.

\subsubsection{Модели со скрытыми переменными}
Все описанные выше подходы требуют хранения разреженной матрицы~$R = \{r_{ui}\}$,
которая может быть достаточно большой.
Более того, они весьма эвристичны и зависят от выбора способа вычисления сходства,
способа генерации товаров-кандидатов, способа их ранжирования.
Альтернативой являются подходы на основе моделей со скрытыми переменными~(latent factor models).

Мы будем пытаться построить для каждого пользователя~$u$ и товара~$i$ векторы~$p_u \in \RR^d$
и~$q_i \in \RR^d$, которые будут характеризовать~<<категории интересов>>.
Например, каждую компоненту такого вектора можно интерпретировать как степень принадлежности
данного товара к определённой категории или степень заинтересованности данного пользователя
в этой категории.
Разумеется, никак не будет гарантироваться, что эти компоненты соответствуют каким-то осмысленным категориям,
если только мы специально не потребуем этого от модели.
По сути, векторы пользователей и товаров являются представлениями~(embeddings),
позволяющими свести эти сущности в одно векторное пространство.

Сходство пользователя и товара будем вычислять через скалярное произведение их представлений:
\[
    r_{ui}
    \approx
    \langle p_u, q_i \rangle.
\]
Также через скалярное произведение можно вычислять сходство двух товаров или двух пользователей.

Мы можем записать функционал ошибки, исходя из способа вычисления сходства:
\begin{equation}
\label{eq:lfm}
    \sum_{(u, i) \in R}
        \left(
            r_{ui}
            - \bar r_u
            - \bar r_i
            - \langle p_u, q_i \rangle
        \right)^2
    \to
    \min_{P, Q}
\end{equation}
Суммирование здесь ведётся по всем парам пользователей и товаров, для которых известен рейтинг~$r_{ui}$.
Заметим, что если~$R^\prime$~--- матрица~$R$ с центрированными строками и столбцами,
то данная задача сводится к низкоранговому матричному разложению:
\[
    \|R^\prime - P^T Q\|^2 \to \min_{P, Q}
\]
Здесь представления пользователей и товаров записаны в столбцах матриц~$P$ и~$Q$.
Существуют модификации, в которых к скалярным произведениям добавляется масштабирующий множитель~$\alpha \in \RR$:
\[
    \|R^\prime - \alpha P^T Q\|^2 \to \min_{P, Q, \alpha}
\]

Данный функционал можно регуляризовать:
\begin{equation}
\label{eq:lfmReg}
    \sum_{(u, i) \in R}
        \left(
            r_{ui}
            - \bar r_u
            - \bar r_i
            - \langle p_u, q_i \rangle
        \right)^2
    +
    \lambda
    \sum_{u \in U}
        \|p_u\|^2
    +
    \mu
    \sum_{i \in I}
        \|q_i\|^2
    \to
    \min_{P, Q}
\end{equation}
Описанная модель носит название Latent Factor Model~(LFM).

Отметим, что использование среднеквадратичной ошибки не всегда имеет смысл~---
в рекомендациях требуется выдать более высокие предсказания для товаров, которые более интересны пользователю,
но вовсе не требуется точно предсказывать рейтинги.
Впрочем, среднеквадратичную ошибку удобно оптимизировать;
более того, именно она использовалась в качестве функционала в конкурсе Netflix Prize,
который во многом определил развитие рекомендательных систем и в котором было предложено много
популярных сейчас методов.

Существует два основных подхода к решению задачи~\eqref{eq:lfm}.
Первый~--- стохастический градиентный спуск, который на каждом шаге случайно выбирает пару~$(u, i) \in R$:
\begin{align*}
    &p_{uk}
    :=
    p_{uk}
    +
    \eta
    q_{ik}
    \left(
        r_{ui}
        - \bar r_u
        - \bar r_i
        - \langle p_u, q_i \rangle
    \right),\\
    &q_{ik}
    :=
    q_{ik}
    +
    \eta
    p_{uk}
    \left(
        r_{ui}
        - \bar r_u
        - \bar r_i
        - \langle p_u, q_i \rangle
    \right).
\end{align*}

Второй подход основан на особенностях функционала~\eqref{eq:lfm} и называется~ALS~(alternating least squares).
Можно показать, что этот функционал не является выпуклым в совокупности по~$P$ и~$Q$,
но при это становится выпуклым, если зафиксировать либо~$P$, либо~$Q$.
Более того, оптимальное значение~$P$ при фиксированном~$Q$~(и наоборот) можно выписать аналитически,~---
но оно будет содержать обращение матрицы:
\begin{align*}
    &p_u
    =
    \left(
        \sum_{i: \exists r_{ui}}
            q_i q_i^T
    \right)^{-1}
    \sum_{i: \exists r_{ui}}
        r_{ui} q_i;\\
    &q_i
    =
    \left(
        \sum_{u: \exists r_{ui}}
            p_u p_u^T
    \right)^{-1}
    \sum_{u: \exists r_{ui}}
        r_{ui} p_u;\\
\end{align*}
(здесь через~$p_u$ и~$q_i$ мы обозначили столбцы матриц~$P$ и~$Q$).

Чтобы избежать сложной операции обращения, будем фиксировать всё, кроме одной строки~$p_k$ матрицы~$P$
или одной строки~$q_k$ матрицы~$Q$.
В этом случае можно найти оптимальное значение для~$p_k$ и~$q_k$:
\begin{align*}
    &p_k
    =
    \frac{q_k (R - \sum_{s \neq k} p_s q_s^T)^T}{q_k q_k^T},\\
    &q_k
    =
    \frac{p_k (R - \sum_{s \neq k} p_s q_s^T)}{p_k p_k^T}.
\end{align*}
Данный подход носит название~Hierarchical alternating least squares~(HALS)~\cite{gillis12hals}.

\subsubsection{Учёт неявной информации}
Выше мы обсуждали, что интерес пользователя к товару может выражаться по-разному.
Это может быть как явный~(выставление рейтинга или лайк, написание рецензии с оценкой),
так и неявный~(просмотр видео, посещение страницы) сигнал.
Неявным сигналам нельзя доверять слишком сильно~--- пользователь мог по многим причинам смотреть
страницу товара.
При этом неявной информации гораздо больше, и поэтому имеет смысл использовать её при обучении моделей.

Один из способов учёта неявной информации предлагается в методе Implicit ALS (iALS)~\cite{hu08ials}.
Введём показатель неявного интереса пользователя к товару:
\[
    s_{ui}
    =
    \begin{cases}
        1, & \exists r_{ui},\\
        0, & \text{иначе}.
    \end{cases}
\]
Здесь мы считаем, что даже если пользователь поставил низкую оценку товару,
то это всё равно лучше ситуации, в которой пользователь совсем не поставил оценку.
Это не очень сильные рассуждения~--- пользователь мог просто не найти
товар, и в таком случае неправильно судить об отсутствии интереса.
Поэтому введём веса~$c_{ui}$, характеризующие уверенность в показателе интереса~$s_{ui}$:
\[
    c_{ui}
    =
    1 + \alpha r_{ui}.
\]
Коэффициент~$\alpha$ позволяет регулировать влияние явного рейтинга на уверенность в интересе.

Теперь мы можем задать функционал:
\[
    \sum_{(u, i) \in D}
        c_{ui} \left(
            s_{ui}
            -
            \bar s_u
            -
            \bar s_i
            -
            \langle p_{u}, q_{i} \rangle
        \right)^{2}
    +
    \lambda
    \sum_{u}
        \|p_u\|^2
    +
    \mu
    \sum_{i}
        \|q_i\|^2
    \to
    \min_{P, Q}
\]

Как и раньше, обучать его можно с помощью стохастического градиентного спуска, ALS или HALS.
Предложенные способы вычисления~$s_{ui}$ и~$c_{ui}$ могут изменяться в зависимости от специфики задачи.

\subsubsection{Факторизационные машины}
Рассмотрим признаковое пространство~$\RR^d$.
Допустим, что целевая переменная зависит от парных
взаимодействий между признаками.
В этом случае представляется разумным строить
полиномиальную регрессию второго порядка:
\[
    a(x)
    =
    w_0
    +
    \sum_{j = 1}^{d}
        w_j x_j
    +
    \sum_{j_1 = 1}^{d}
    \sum_{j_2 = j_1 + 1}^{d}
        w_{j_1 j_2}
        x_{j_1} x_{j_2}.
\]
Данная модель состоит из~$d (d - 1) / 2 + d + 1$ параметров.
Если среди признаков есть категориальные с большим числом категорий~(например,
идентификатор пользователя), то после их бинарного кодирования число параметров
станет слишком большим.
Чтобы решить проблему, предположим, что вес
взаимодействия признаков~$j_1$ и~$j_2$ может быть
аппроксимирован произведением низкоразмерных скрытых векторов~$v_{j_1}$ и~$v_{j_2}$,
характеризующих эти признаки.
Мы получим модель, называемую~\emph{факторизационной машиной}~(factorization machine, FM)~\cite{rendle12fm}:
\[
    a(x)
    =
    w_0
    +
    \sum_{j = 1}^{d}
        w_j x_j
    +
    \sum_{j_1 = 1}^{d}
    \sum_{j_2 = j_1 + 1}^{d}
        \langle v_{j_1}, v_{j_2} \rangle
        x_{j_1} x_{j_2}.
\]
Благодаря описанному трюку число параметров снижается до~$dr + d + 1$,
где~$r$~--- размерность скрытых векторов.

Данная модель является обобщением моделей с матричными разложениями.
Задачу~\eqref{eq:lfm} можно сформулировать как задачу построения регрессии
с двумя категориальными признаками: идентификатором пользователя
и идентификатором товара.
Целевым признаком является рейтинг~$r_{ui}$.
Для некоторого подмножества пар (пользователь, товар)
мы знаем рейтинг; для остальных мы хотим его восстановить.
После бинаризации признаков получим, что каждый объект~$x$ описывается~$|U| + |I|$
признаками, причём ненулевыми являются ровно два из них: один соответствует
номеру пользователя~$u$, второй~--- номеру товара~$i$.
Тогда факторизационная машина примет следующий вид:
\[
    a(x)
    =
    w_0
    +
    w_{u}
    +
    w_{i}
    +
    \langle v_u, v_i \rangle.
\]
Данная форма полностью соответствует модели~\eqref{eq:lfm}.
%После бинаризации признаков мы получим, что факторизационная машина
%оценивает целевую переменную как произведение
%скрытых векторов пользователя и товара~--- иными словами,
%она строит разложение матрицы рейтингов~$R$.
По сути, факторизационная машина позволяет строить рекомендательные модели
на основе большого количества категориальных и вещественных признаков.

Существует несколько методов настройки факторизационных машин,
из которых наиболее совершенным считается метод Монте-Карло
на основе марковских цепей;
реализацию можно найти в библиотеке libFM.

\paragraph{FFM.}
Недавно было предложено расширение факторизационных машин,
позволившее авторам победить в конкурсах Criteo и Avazu
по предсказанию кликов по рекламным объявлениям.
В обычных факторизационных машинах у каждого признака
имеется всего один скрытый вектор, отвечающий за взаимодействие
с остальными признаками.
Допустим, что признаки можно некоторым образом сгруппировать~---
например, в задаче рекомендации музыкальных альбомов в бинарном векторе,
отвечающем за композиции, будет стоять несколько единиц, соответствующих
всем композициям из альбома.
Все единицы из этого вектора можно объединить в одну группу.
Расширим модель, введя для каждого признака разные скрытые векторы
для взаимодействия с разными группами:
\[
    a(x)
    =
    w_0
    +
    \sum_{j = 1}^{d}
        w_j x_j
    +
    \sum_{j_1 = 1}^{d}
    \sum_{j_2 = j_1 + 1}^{d}
    \langle v_{j_1, f_{j_2}}, v_{j_2, f_{j_1}} \rangle
        x_{j_1} x_{j_2},
\]
где~$f_{j_1}$ и~$f_{j_2}$~--- индексы групп признаков~$x_{j_1}$ и~$x_{j_2}$.
Данная модель носит название~\emph{field-aware factorization machines}~(FFM)~\cite{ffm}.

\subsection{Контентные модели}
В коллаборативной фильтрации используется информация о предпочтении пользователей и об их сходствах,
но при этом никак не используются свойства самих пользователей или товаров.
При этом может быть полезно находить товары, которые своим описанием похожи на товары из историю пользователя;
особенно релевантно это может быть для рекомендательных систем контента~(музыки, статей, видео),
где пользователю, скажем, захочется познакомиться с музыкой, похожей на музыку
его любимых исполнителей.

Как правило, это приводит к следующей идее: все товары описываются с помощью векторов~(представлений, embeddings),
и затем измеряется сходство между вектором нового товара и векторами товаров из истории пользователя.
Можно вычислять минимальное или среднее расстояние до векторов из истории.
Можно обучить линейную модель, которая для данного пользователя предсказывает целевую переменную на основе
представления товара:
\[
    \sum_{i \in I: \exists r_{ui}}
    \left(
        \langle w_u, q_i \rangle
        -
        r_{ui}
    \right)^2
    \to
    \min_{w_u},
\]
и затем с помощью этой модели оценивать, насколько пользователю подойдут другие товары.
Можно обучить граф вычислений, который по всем данным о товаре и о пользователе пытается предсказать
целевую переменную.
Существует много методов, и какой из них подойдёт для данной задачи~--- заранее предсказать нельзя.

\subsection{Статистические признаки}
Важны и более простые типы факторов: конверсия просмотра данного товара в покупку за всю историю магазина,
число покупок данного пользователя в категории данного товара, число покупок данного пользователя и т.д.
Если товар или пользователь уже набрали достаточно статистики, то зачастую такие признаки
оказываются самыми главными при принятии решения, поскольку уже содержат в себе достаточно информации
о предпочтениях.


\begin{thebibliography}{1}
\bibitem{gillis12hals}
    \emph{Gillis, Nicolas and Glineur, Fran\c{c}ois} (2012).
    Accelerated Multiplicative Updates and Hierarchical Als Algorithms for Nonnegative Matrix Factorization.~//
    Neural Comput., 24, 4, p. 1085--1105.
\bibitem{hu08ials}
    \emph{Hu, Yifan and Koren, Yehuda and Volinsky, Chris} (2008).
    Collaborative Filtering for Implicit Feedback Datasets.~//
    ICDM '08.
\bibitem{rendle12fm}
    \emph{Rendle, S.} (2012).
    Factorization machines with libFM.~//
    ACM Trans. Intell. Syst. Technol. 3, 3, Article 57.
\bibitem{ffm}
    Field-aware Factorization Machines: \\\url{http://www.csie.ntu.edu.tw/\~r01922136/slides/ffm.pdf}
\bibitem{shani11eval}
    \emph{Guy Shani, Asela Gunawardana} (2011).
    Evaluating recommendation systems.~//
    Recommender systems handbook, pp. 257-297. Springer.
\end{thebibliography}



\end{document}
